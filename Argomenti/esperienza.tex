\chapter{Esperienza di laboratorio}
\section{Acceleratori allo stato solido}
	Negli smartphone sono presenti accelerometri \textit{low-cost} che si basas sulla micro-lavorazione del silicio; di per se il sensore non sarebbe \textit{buono}, tuttavia l'effetto dovuto all'elevato guadagno in retroazione si \textit{migliora} il comportamento del segnale.
	
	\textbf{SCHEMA}
	
	Osservando lo schema è possibile capire come l'accelerazione permette di spostare  il piatto centrale: questo cambia la capacità relativa tra le due armature fisse e dunque è possibile stimare il misurando. 
	
	Per migliorare la misura ci si aiuta con dell'elettronica che, tramite retroazione, tende ad evitare lo spostamento della massa centrale (e dunque la parte meccanica di \textit{scarsa qualità} associata al silicio) e dunque la stima dipende dalla misura della capacità, che può essere effettuata con buona precisione. La misura viene effettuata considerando che la forza dovuta alle armature elettriche che deve essere impressa al corpo è proporzionale all'accelerazione.
	
	
	