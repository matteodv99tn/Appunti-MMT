\chapter{Misura di spostamento}

\section{Encoder ottici}

\section{Potenziometro}
	Il \de{potenziometro} è un trasduttore della misura ingresso-spostamento che viene trasdotta in una tensione elettrica; il moto dei corpi Potenziometroviene caratterizzato dalle grandezze di spostamento, velocità e accelerazione (le ultime possono essere ricavate per derivazione). Essi permettono inoltre di misurare degli spostamenti sia assoluti che relativi, ma anche di tipo lineare o angolare.
	
	I \textbf{potenziometri} possono essere a \de{traslazione} che si basano sulla traslazione di un cursore, il potenziometro a \de{rotazione ad un giro} (analogo a quello di traslazione che tuttavia misura variazioni angolari) e \de{a rotazione ad elica} (permette di misurare più giri).
	
	\paragraph{Potenziometro rotativo} \textbf{FIGURA}
	
	Un potenziometro rotativo è un componente a 3 \textit{pin}: i due più esterni sono collegati alla tensione di alimentazione, mentre il pin centrale presenta la tensione in uscita ottenuta dal partitore di tensione della resistenza interna del potenziometro.
	
	La resistenza interna è tendenzialmente realizzata da polveri ceramiche/plastiche. Il modello ideale del potenziometro è dunque quello del partitore di tensione la cui somma delle resistenze è pari alla resistenza nominale $R_p$. La ripartizione della resistenza complessiva dipende dunque dalla posizione del cursore dipendente dal misurando del sistema.
	
	Nota la lunghezza $L$ della corsa e la posizione $x$ del cursore, allora il partitore è composto dalle resistenze
	\[ R_1 = R_p \frac{L-x}{L} \qquad R_2 = R_p \frac x L\]
	Dall'equazione del partitore (dove $V_{dd}$ è la tensione di alimentazione del circuito) è possibile scrivere la tensione in uscita nel caso ideale come 
	\begin{equation} \label{eq:pot:ideale}
		V_{out}= V_{dd} \frac{R_2}{R_1+R_2} = V_{dd} \frac{x}{L-x+x} = V_{dd} \frac x L
	\end{equation}
	
	Il modello ottenuto è lineare, tuttavia facendo la taratura di un sistema reale si osserva uno scostamento dal caso ideale che può essere tuttavia modellato da una resistenza $R_m$ che rappresenta il carico esterno (dovuta all'oggetto che effettua la misurazione della tensione).
	
	\textbf{INSERIRE SCHEMA MODELLO REALE} 
	
	A questo punto il partitore di tensione è dato dalla resistenza $R_1$ e dal parallelo di $R_2$ con $R_m$ pari a 
	\[ R_2 \parallel R_m = \frac{R_mR_p \frac x L}{R_m + R_p \frac x L} \qquad\]
	Questo permette di ottenere un nuovo modello che considera anche la resistenza di carico $R_m$ come
	\[ V_{out} = V_{dd} \frac{R_2\parallel R_m}{ R_1 + R_2\parallel R_m} = V_{dd} \frac{x/L}{1 + \dfrac{R_p}{R_m} \dfrac{x}{L} \left(1 - \dfrac x L\right)   } \]
	A questo punto è possibile osservare che un modello più realistico non è più lineare (rispetto al misurando $x$) come il modello ottenuto in prima battuta in equazione \ref{eq:pot:ideale}. In particolare nel caso in cui $R_m \rightarrow \infty$ (resistenza di carico infinita) allora il comportamento risulterebbe essere approssimato al caso ideale con caratteristica di uscita lineare. Un altro modo per ottenere valori reali \textit{simili} al comportamento ideale sarebbe quello di avare una resistenza complessiva $R_p$ tendente a zero (richiesta che si scontra con la realizzazione pratica dei potenziometri reali).
	
	\textbf{FIGURA DEL COMPORTAMENTO REALE}
	
	In particolare il problema di scegliere un valore $R_p$ si ottiene una sensibilità bassa in quanto sarebbe limitata  la tensione $V_{dd}$ di alimentazione del meccanismo. Ogni resistenza può infatti dissipare una potenza massima $P$, ed essendo che la caduta di tensione alla stessa $V$ allora si ricava che
	\[P = \frac{V^2}{R_p} \qquad \Rightarrow \quad V = \sqrt{PR_p}\]
	Dunque se $R_p$ piccolo, allora necessariamente $V_{dd}$ è bassa: questo porta ad un'abbassamento della pendenza della caratteristica di uscita che diminuisce la sensibilità del sistema. Inoltre il problema è anche termico in quanto se la potenza da dissipare è eccessiva, il dispositivo potrebbe andare incontro a rottura.
	
	\textbf{FARE TABELLA DEL PERCHÈ SI SCEGLIE RP}
	
	\paragraph{Sensibilità al contatto} Un possibile problema del funzionamento del potenziometro è quello dovuto alla sensibilità del contatto dovuto alle vibrazioni meccaniche. Per eliminare questo problema è possibile immergere il cursore nell'olio (che funge da smorzatore di oscillazioni) oppure equipaggiare il potenziometro con due testine con modi di vibrazione disaccoppiati.
	
	
	
	
	
	
	
	
	
	
	
	
	
	
	
	
	
	
	
	