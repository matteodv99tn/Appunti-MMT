\chapter{Termocoppie e termoresistenze}
	La misura di temperatura avviene misurando la differenza di forza elettromotrice (tensione) di una termocoppia
	
	\paragraph{Effetto Peltier} Quando una corrente inizia a scorrere nella termocoppia, sui giunti metallici si genera del calore (o esso viene dissipato) in modo da controbilanciare l'effetto Seebeck, anche se spesso questo effetto viene trascurato.
	
	\paragraph{Effetto Thomson} L'effetto Thomson rileva la dispersione di calore dovuta al filo che collega le due termocoppie.
	
	In generale
	
	\[ fem_{termocoppia} = \underbrace{c_1 \big(T_1-T_2\big)}_\textrm{Seebeck} +\underbrace{ c_2\big(T_1^2-T_2^2\big)}_\textrm{Thomson}\]
	dove $c_1,c_2$ sono due costanti generalmente fornite dal produttore ma che devono essere tarate per migliorare la precisione di misura. In generale i fili di connessione delle due giunzioni possono essere esposti a temperature ambientali variabili in quanto non influiscono nella misurazione della forza elettromotrice della termocoppia.
	
	Nella pratica le termocoppie sono caratterizzate dal non avere la necessità di avere un ambiente termostatato, ma si integra il controllore con un sensore di temperatura con il quale fare il confronto di temperatura; il sistema si comporta come un termometro dunque.
	
	
\section{Termoresistenze}
	Le termoresistenze si basano sul fatto che la resistenza $R$ di un metallo è funzione delle temperatura $T$ cui esso è sottoposto; per esempio è possibile calcolare la resistenza in funzione dell'intervallo di temperatura secondo le leggi
	\[ R(T_{^\circ C}) = \begin{cases}
		R_0\big(1 + aT + bT^2 + c(T-100)^3\big) \qquad & T\in [-200^\circ C, 0^\circ C] \\
		R_0\big(1 + aT + bT^2\big) \qquad &T\in [0^\circ C,850^\circ C] 
	\end{cases}\]
	dove la temperatura $T$ è espressa in gradi Celsius, $R_0$ è la resistenza del conduttore alla temperatura $0^\circ C$ e $a,b,c$ sono dei coefficienti del metallo.
	
	Per la realizzazione di termoresistenze in generale si utilizzano nickel, rame e platico; in particolare noi facciamo riferimento alla sonda \texttt{PT100} realizzata in platino con resistenza $R_0 = 100 \Omega$ con classe di accuratezza $A$ (più accurata) e $B$ (meno accurata).
	
	Per disaccoppiare dall'effetto di temperatura quello di deformazione si avvolge il filo a spirale e si immerge il metallo in una polvere di allumina che isola dalle vibrazioni pur conducendo calore (in questo modo si elimina l'effetto \textit{estensimetro}).
	
	Trascurando la resistenza dei fili di collegamento tra termoresistenza e apparato di misurazione, la temperatura viene misurata tramite un ponte di Wheatstone semplice. Il trascurare i cavi può portare a grossi errori in quanto aumentano la resistenza equivalente del circuito e la temperatura dei fili può modificare la temperatura che si vuole misurare. Per ovviare a questo problema, come per gli estensimetri, si effettua un collegamento a tre fili con il ponte di Wheatstone
	
	\textbf{AGGIUNGERE LE FIGURE} 
	
	Un altro problema del circuito è dovuto alla bassa resistenza della sonda; in generale dovendo essere pari a $20mA$ la corrente massima che può scorrere sull circuito, considerando le due resistenze $R_1,R_2= 100\Omega$ la tensione di alimentazione è limitata al valore
	\[ V= I_{max} R_{eq} = 4V \]
	
	Un modo per ovviare a questo problema è di porre a $25V$ la tensione di alimentazione da modulare con un duty cycle in modo che la tensione media nel tempo è di $4V$.
	
	Un altro metodo di misurazione è quello \textbf{volt-amperometrico} con collegamento a due fili; in questo caso il circuito è alimentato da un generatore di corrente e, tramite un multimetro, si misura la differenza di tensione ai capi dei due fili. In questo caso l'uscita non è proporzionale solamente alla resistenza della sonda, ma anche a quella dei fili. \\
	Per ridurre questo problema si utilizza lo stesso metodo ma a 4 fili
	
	
	
	
	
	
	
	
	
	
	