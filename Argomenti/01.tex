\chapter{Introduzione}
	\de{Misurare}
	
	
	
\section{Teoria generalizzata degli strumenti di misura}
	Gli \de{elementi funzionali} individualibili un un apparato (o sistema) di misura:
	\begin{enumerate}
		\item il \textbf{sensore primario}: l'elemento sensibile al misurando. E' l'elemento in diretto contatto con il misurando (o è preceduto da un elemento trasmettitore). Esso riceve energia dal misurando e, a volte, può effettuare una trasformazione di variabile. In generale è importante che questo sensore perturbi il meno possibile il misurando, assorbendo il minimo di energia;
		
		\item l'\textbf{elemento di trasmissione}: è un elemento un cui la grandezza di ingresso è pari a quella in uscita ma opera una trasduzione di posizione del segnale utile. In particolare esso deve attenuare, distorcere e ritardare il meno possibile il segnale condotto;
		
		\item l'\textbf{elemento di conversione di grandezza} o \textbf{trasformatore di variabile}: è utilizzato per facilitare la trasmissione del segnale a distanza. Può anche essere utilizzato per elaborare il segnale (come amplificarlo o filtrarlo). In generale si ha la trasformazione in questo punto in grandezze elettriche;
		
		\item\textbf{ elemento di elaborazione dati}: utilizzato per elaborare i dati come la codifica dei segnali per la trasmissione, per amplificare ulteriormente i segnali, estrarre informazioni aggiuntive o ridurre l'effetto degli ingressi di disturbo. Si ha elaborazione anche quando si integra/deriva una variabile per ottenere informazioni ulteriori o convertire segnali da analogico a digitale (e viceversa);
		
		\item \textbf{elemento di memorizzazione}: servono a memorizzare (a breve o lungo termine) le informazioni provenienti da un sistema di misura. Essi permettono di cambiare anche la scala dei tempi di riproduzione dei dati;
		
		\item \textbf{elemento di presentazione dati}: fornisce l'uscita dei dati in una forma reattiva ai sensi dell'uomo (tendenzialmente segnali visivi).
	\end{enumerate}	