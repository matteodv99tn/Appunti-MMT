\chapter{Misure di deformazione mediante estensimetri}

\section{Cenni sullo stato di tensione piano e triassiale}
	Considerando uno \de{stato di tensione piano} (tensione perpendicolare al piano $\sigma_z=0$), è possibile determinare le deformazioni $\epsilon_x, \epsilon_y$ lungo gli assi $x$ e $y$ rispettivamente supponendo di conoscere le tensioni $\sigma_x,\sigma_y$ lungo gli stessi. Noto il \de{modulo di Young} $E$ e il \de{coefficiente di Poisson} $\nu$ si verifica infatti che
	\begin{equation} \label{eq:def:defpiana}
		\epsilon_x = \frac{\sigma_x}{E} - \nu \frac{\sigma_y}{E} \qquad \epsilon_y = \frac{\sigma_y}{E} - \nu \frac{\sigma_x}{E}
	\end{equation}
	
	Le deformazioni $\epsilon$ rappresentano una variazione di lunghezza $\Delta L$ riferite ad una lunghezza $L$, e in generale si considera
	\[ \epsilon = \frac{\Delta L}{L} \qquad \left[\frac {\mu m} m \right]\]
	
	Invertendo opportunamente le formule è possibile determinare le tensioni $\sigma_i$ lungo i vari assi del piano come
	\[ \sigma_x = \frac{E \big(\epsilon_x + \nu \epsilon_y \big)}{1-\nu^2} \qquad \sigma_y = \frac{E \big(\epsilon_y + \nu \epsilon_x \big)}{1-\nu^2} \]
	
	\paragraph{Stato di tensione triassiale} Estendendo i concetti della deformazione di uno stato piano allo spazio, è possibile definire la deformazione $\epsilon$ lungo gli assi nello spazio riprendendo l'equazione \ref{eq:def:defpiana} riscritta come
	\begin{equation}
		\epsilon_x = \frac{\sigma_x}{E} - \nu \frac{\sigma_y}{E} - \nu \frac{\sigma_z}{E} \qquad \epsilon_y = \frac{\sigma_y}{E} - \nu \frac{\sigma_x}{E} - \nu \frac{\sigma_z}{E} \qquad \epsilon_z = \frac{\sigma_z}{E} - \nu \frac{\sigma_x}{E} - \nu \frac{\sigma_y}{E}
	\end{equation}

\section{Classificazione degli estensimetri}
	Gli \de{estensimetri} sono utilizzati per determinare deformazioni superficiali locali (quindi analizzando punti molto vicini tra loro). Sono detti invece \de{estensometri} i componenti atti a determinare deformazioni medie superficiali su punti \textit{distanti} tra loro.
	
	Gli estensimetri possono essere:
	\begin{itemize}
		\item meccanici, i più antichi. Sono composti da due coltelli (uno fisso e uno mobile) che tramite delle leve meccaniche permettono di amplificare il loro spostamento relativo permettendo di visualizzare con precisione l'estensione dei punti oggetti di studio;
		
		\item acustici;
		\item ottici;
		\item elettrici a resistenza; comunemente i più usati attualmente.
	\end{itemize}

\section{Estensimetro a resistenza elettrica}
	\subsection*{Caratteristiche desiderate}
		Le caratteristiche desiderate in ogni estensimetro sono:
		\begin{itemize}
			\item che la costante di taratura sia stabile e tempo invariante, sia per effetti termici che altri fattori ambientali;
			\item deve misurare la deformazione locale e non quella media, quindi deve riuscire ad analizzare punti molto vicini tra loro;
			\item deve avere  una buona risposta in frequenza per rilevare deformazioni che avvengono molto velocemente;
			\item devono essere economicamente accessibili per favorire un largo impiego.
		\end{itemize}
	
	\subsection{Principio di funzionamento}
		Il funzionamento di un estensimetro a resistenza $R$ si basa sul fatto che tale proprietà risulta essere proporzionale all'allungamento del filo rispetto alla quale è ricavato. In particolare disponendo un filo a maglia (quindi la lunghezza complessiva è elevata), applicando ai suoi terminali una tensione costante è possibile determinare la resistenza $R$ del circuito fortemente determinata dall'allungamento.
		
		Estensimetri elettrici presentano tendenzialmente resistenze interne pari a $R=120\Omega,350\Omega$. Ogni soluzione presenta dei vantaggi ma anche degli svantaggi, come si può osservare in tabella \ref{tab:def:vantaggi}. Questo tipo di estensimetri permettono di misurare punti inizialmente posti tra $0.6mm$ e $200mm$.
		
		\begin{SCtable}[0.5][b!]
			\centering
			\begin{tabular}{p{5cm} p{1.2cm} p{1.2cm} }
				 &   $120\Omega$ &  $350\Omega$ \\ \hline
				 corrente di alimentazione &   alta &   bassa \\ \hline
				 autoriscaldamento &   alto &   basso \\ \hline
				 sensibilità alla riduzione dell'isolamento &   bassa &   alta \\ \hline
				 sensibilità all'interferenza elettromagnetica &   bassa &   alta \\ \hline
				 effetto di resistenze nella trasmissione (cavi) &   alto &   basso \\
			\end{tabular}
			\caption{vantaggi e svantaggi associati alle resistenze iniziale degli estensimetri elettrici.} \label{tab:def:vantaggi}
		\end{SCtable}
		
	\subsubsection{Dimostrazioni matematiche}
		Per un conduttore filiforme di lunghezza $l$ e sezione $A$, allora la resistenza $R$ dello stesso è pari a 
		\begin{equation}
			R= \rho \frac l A
		\end{equation}
		dove $\rho$ è la \de{resistività} del materiale, ossia una proprietà intrinseca del metallo che compone il conduttore. Per determinare la variazione di resistenza $\delta R$ rispetto alla resistenza $R$ è necessario differenziare tale equazione, con il risultato che
		\begin{align*}
			\frac{\delta R}{R} & = \frac 1 R \left( \pd R \rho \delta \rho + \pd R l \delta l + \pd R A \delta A \right) \\ 
			& = \frac{\delta \rho}{\rho} + \frac{\delta l}{l} - \frac{\delta A}{A}
		\end{align*}		
		
		In generale l'area $A$ della sezione del filo può essere espressa, tramite un fattore di forma $C$ costante, come proporzionale al quadrato del diametro $d$ del conduttore, e dunque
		\[ A = C d^2 \qquad \rightarrow \frac{\delta A}{A} = \cancel{\pd A C\delta C}+ \pd A d \delta d = 2\frac{Cd \, \delta d}{A} = 2\frac{\delta d}{d}\]
		A questo punto la variazione della resistenza del conduttore può essere espressa come 
		\[ \frac{\delta R}{R} =  \frac{\delta \rho}{\rho} + \frac{\delta l}{l} - 2\frac{\delta d}{d} \]
		
		Nota la deformazione assiale $\epsilon_a$ (in quanto è quella che determina la misura di deformazione), allora ad essa è associata anche una deformazione trasversale $\epsilon_t = - \nu \epsilon_a$. Considerando che $\epsilon_a$ rappresenta il termine $\delta l/l$ l'espressione della resistenza può essere riscritta come:
		\begin{align*}
			\frac{\delta R}{R} =  \frac{\delta \rho}{\rho} + \frac{\delta l}{l} - 2\left( -\nu \frac{\delta l}{l} \right) & = \frac{\delta \rho}{\rho} + \frac{\delta l}{l} \big(1+2\nu\big) \\ 
			& = \epsilon_a \frac{\delta \rho}{\rho \epsilon_a} + \epsilon_a \big(1+2\nu\big) = \epsilon_a \left( \frac{\delta \rho}{\rho \epsilon_a} +  1+2\nu\right)
		\end{align*}
		
		\begin{concetto}
			A questo punto è possibile esprimere la \de{prima legge fondamentale dell'estensimetria} che permette di definire la \de{sensibilità dell'estensimetro} $K$:
			\begin{equation} \label{eq:def:primalegge}
				K = \frac{\delta R / R}{\epsilon_a} = \frac{\delta \rho}{\rho \epsilon_a} + 1 + 2\nu
			\end{equation} 
		\end{concetto} 
	
		Il termine $1+2\nu$ rappresenta la \de{sensibilità geometrica} ed è sempre un valore costante (essendo $\nu = 0.3$ per i metalli, allora essa vale circa $1.6$), mentre il termine $\delta \rho / \rho \epsilon_a$ rappresenta la \de{sensibilità piezoresistiva} (circa 0.4 per estensimetri metallici, tra 100 e 1000 per quelli a semiconduttore). Si osserva che questo secondo termine dipende esplicitamente dalla deformazione $\epsilon_a$, e dunque può essere trascurato per piccole deformazioni, ma può diventare importante per estensioni maggiori.
		
		Con questa legge è dunque possibile correlare resistenza elettrica ed estensione come
		\[ \frac {\delta R} R = K \epsilon_a  \]
		
		\paragraph{Sensibilità trasversale} Le relazioni appena indicate non considerano il fatto che gli estensimetri contengano tratti di griglia perpendicolari all'asse di misura che subiscono delle deformazioni differenti $\epsilon_t$. Tuttavia questi tratti sono pochi e di dimensioni limitati, e per questo spesso è convenienti trascurarli. 
		
		Considerando che la sezione trasversale ha una sensibilità $K_t$, è possibile determinare il \de{coefficiente di sensibilità trasversale} $S_t$ come segue che permette di ridefinire l'equazione di variazione di resistenza:
		\[ S_t = \frac{K_t}{K} \qquad \rightarrow \frac {\delta R} R = K\big(\epsilon_a+ S_t\epsilon_t\big)  \]
		
		Considerando che in generale $S_t < 1\%$ e che $\epsilon_t<\epsilon_a$, trascurando il termine dovuto alla sensibilità trasversale non comporterebbe l'introduzione di un errore significativo (per piccole deformazioni).
		
	\subsection{Lettura degli estensimetri}
	\subsubsection{Ponte di Wheatstone}
		Per interpretare le variazioni di resistenza degli estensimetri si ricorre all'utilizzo del \de{ponte di Wheatstone} (figura \ref{wheatstone}), un circuito particolare di elementi resistivi che permette di determinare lo \de{sbilanciamento} $e_0$ del ponte.
		
		\figura{5}{0.75}{wheatstone}{circuito del ponte di Wheatstone.}{wheatstone}
		
		Alimentando ad una tensione costante $E$ il circuito, tramite le equazioni di Kirchhoff è possibile determinare delle relazioni algebriche tra sbilanciamento $e_0$ e tensioni di alimentazione. Considerando il sistema di equazioni:
		\[ \begin{cases}
			-e_0 - R_2I_2 +R_1I_1 = 0 \\ E = (R_1+R_4) I_1 \\ E = (R_2+R_3)I_2
		\end{cases}\]
		e' possibile determinare il rapporto $e_0/E$ come:
		\[ \frac{e_0}{E} = \frac{R_1}{R_1+R_4} - \frac{R_2}{R_2+R_3} = \frac{R_1R_3-R_2R_4}{\big(R_1+R_4\big) \big(R_2+R_3\big)} \]
		
		Supponendo di avere il ponte inizialmente azzerato, allora deve valere che $e_0 = R_1R_3-R_2R_4=0$. Considerando che la resistenza $R_1$ vari di una quantità $\Delta R_1$ (assumendo valore complessivo $R_1+\Delta R_1$), allora è possibile (tramite la relazione mostrata precedentemente) determinare la variazione di sbilanciamento $\Delta e_0$ del ponte come:
		\begin{align*}
			\frac{\Delta e_0}{E} & = \frac{\big(R_1+\Delta R_1\big)R_3-R_2R_4} {\big(R_1+\Delta R_1+R_4\big)\big(R_2+R_3\big)} = \frac{\Delta R_1 R_3}{\big(R_1+R_4\big) \big(R_2+R_3\big) + \Delta R_1 \big(R_2+R_3\big)} \\
			 &= \frac{\Delta R_1 R_3}{\big(R_1+R_4\big) \big(R_2+R_3\big) } \frac{1}{1+\frac{\Delta R_1}{R_1+R_4}}
		\end{align*}
		
		Introducendo i parametri $\alpha= R_3 / (R_2+R_3)$ e $R_m = R_1+R_4$ l'espressione precedente può essere riscritta in forma più compatta come
		\[ \frac{\Delta e_0}{E} = \frac{\Delta R_1 \, \alpha}{R_m} \frac{1}{1+\Delta R_1/R_m} \]
		Nell'ipotesi di considerare $\Delta R_1/R_1$ molto piccolo e $R_1=R_2=R_3=R_4$ allora l'equazione precedente può essere approssimata a
		\[ \frac{\Delta e_0}{E} = \frac{\Delta R_1 \, \alpha}{R_m} \cancel{\frac{1}{1+\Delta R_1/R_m}} \approx \frac 1 2 \frac{\Delta R_1}{2R_1} = \frac 1 4 \frac{\Delta R_1}{R_1} \]
		Con questa formulazione è dunque possibile osservare che lo sbilanciamento del ponte $\Delta e_0$ è direttamente proporzionale alla variazione di resistenza $\Delta R_1$. Utilizzando il principio di sovrapposizione degli effetti è dunque possibile determinare la variazione di sbilanciamento nel caso in cui ognuna delle resistenze cambi di valore (sempre nell'ipotesi in cui tutte le resistenze abbiano valore nominale costante):
		\[ \frac{\Delta e_0}{E} \approx \frac 1 4 \left( \frac{\Delta R_1}{R_1} - \frac{\Delta R_2}{R_2} +\frac{\Delta R_3}{R_3} - \frac{\Delta R_4}{R_4} \right) \qquad \Leftrightarrow \qquad \frac{\Delta R_i}{R_i} < 1\% \]
		
		\begin{concetto}
			Ricordando che dalla prima legge fondamentale dell'estensimetria (eq. \ref{eq:def:primalegge}) $K\epsilon = \Delta R / R$, allora è possibile scrivere la \de{seconda legge fondamentale dell'estensimetria} che lega lo sbilanciamento $e_0$ del ponte di Wheatston agli allungamenti di 4 estensimetri:
			\begin{equation}
				\frac{\Delta e_0}{E} \approx \frac 1 4 K \big( \epsilon_1 - \epsilon_2 + \epsilon_3 - \epsilon_4  \big)
			\end{equation}
		\end{concetto}
		
		In base al numero di estensimetri (che dunque possono essere modellati come dei resistori a resistenze variabili) è possibile ottenere sistemi ad un \textbf{quarto di ponte} (un solo estensimetro), a \textbf{mezzo ponte} (due estensimetri sullo stesso ramo) o a \textbf{ponte intero} (quattro estensimetri, per ogni resistenza).
		
		\paragraph{Errori dovuti alle approssimazioni} Si è arrivati a questo risultato considerando l'approssimazione
		\[\frac{\Delta R_1 \, \alpha}{R_m} \frac{1}{1+\Delta R_1/R_m} \approx \frac{\Delta R_1}{R_m} \]
		Questa approssimazione deriva dal fatto che $\Delta R_1/R_m < 1\% = 10^{-2}$ può essere considerato nullo e dunque $1/1=1$. Per poter stabilire tuttavia oltre quale tensione tale errore non è più tollerabile è necessario imporre (considerando $R_m = 2R_1$)
		\[ \frac{\Delta R_1}{2R_1} = K \frac\epsilon 2 > 10^{-2} \qquad \Rightarrow \quad \epsilon> \frac 1 K 2\cdot 10^{-2} \frac m m \simeq 10^4 \frac{\mu m}{m}  \]
		
	\subsection{Effetti termici}
		In generale la temperatura ambientale può influire sulla misura di deformazione in quanto varia direttamente la resistività del filo, la lunghezza della griglia (per via della deformazione dovuta alla temperatura) modificando direttamente la \textbf{sensibilità piezoresistiva}; in particolare si era ricavato dalla prima equazione fondamentale che la sensibilità dell'estensimetro vale
		\[ K = 1 + 2\nu + \frac{\delta \rho / \rho}{\epsilon}\]
		
		\paragraph{Variazione di resistività} Una variazione di temperatura $\Delta T = T - T_0$ influenza direttamente la resistività del materiale (dove la condizione a pedice 0 è nota da laboratorio) secondo la legge:
		\[ \Delta \rho = \rho_T - \rho_0 = \rho_0 \alpha_\rho \, \Delta T  \]
		dove $\alpha_\rho$ è il coefficiente termico per la resistività. A questo punto è possibile calcolare la variazione di resistenza tramite la resistività cambiata dalla temperatura è pari a 
		\[ \Delta R_\rho = R_\rho(T) - R_\rho(T_0) = R_0\alpha_\rho \, \Delta T  \]
		
		\paragraph{Variazione di lunghezza} La temperatura determina una deformazione di lunghezza $\Delta l$ dell'estensimetro a causa della dilatazione del materiale su cui è incollato l'estensimetro (coefficiente di dilatazione termica $\alpha_m$) e quello del materiale di cui è costituita la griglia (coefficiente di dilatazione $\alpha_g$):
		\[  \Delta l = l \big(\alpha_m - \alpha_g\big) \, \Delta T \qquad \Rightarrow \quad \epsilon_{\Delta T} = \frac{\Delta l}{l} = \big(\alpha_m-\alpha_g\big) \, \Delta T  \]
		
		Osservando che alla temperatura è associato un allungamento $\epsilon_{\Delta T}$ è possibile misurare la variazione di resistenza dovuta alla temperatura come
		\[ \frac{\Delta R}{R_0} = K \big(\alpha_m-\alpha_g\big)\, \Delta T \qquad \rightarrow \Delta R = R_0 K \big(\alpha_m - \alpha_g\big) \, \Delta T   \]
		
		\paragraph{Combinazione degli effetti termici} Combinando gli effetti termici è possibile determinare la variazione di resistenza $\Delta R_{\Delta T}$ dovuta alla temperatura come
		\[ \Delta R_{\Delta T} = \Delta R_\rho + \Delta R = R_0 \alpha_\rho \, \Delta T + R_0 K \big(\alpha_m - \alpha_g\big) \, \Delta T  \]
		\[ \Rightarrow \qquad \frac{\Delta R_{\Delta T}}{R_0} = \Delta T \Big[ \alpha_\rho + K \big(\alpha_m-\alpha_g\big)\Big]  \]
		Scegliendo opportunamente i materiali dell'estensimetro e della griglia tali che $\alpha_\rho + K(\alpha_m-\alpha_g)$ tenda a zero, è possibile auto-compensare l'effetto termico riducendo di fatto l'effetto interferente.
		
		\paragraph{Effetto modificante} In generale la variazione della sensibilità piezoresistiva può essere trascurata; considerando infatti la costantana (metallo solitamente utilizzato negli estensimetri) si considera che per una temperatura ambiente $T_{amb} = 24^\circ C$, nel range di temperatura $\pm 25^\circ C$ la correzione da apportare al fattore di taratura è minore del $0.5\%$: in generale tale incertezza è dello stesso ordine di grandezza dell'incertezza con cui è noto tale parametro.
		
		Volendo utilizzare estensimetri elettrici per temperature diverse i produttori degli stessi forniscono generalmente delle equazioni che permettono di determinare la sensibilità $K$ in funzione di un parametro $\beta$ e della temperatura di riferimento $T_0$ al variare della temperatura $T$ seguendo una legge
		\[ K(T) = K(T_0) \, \Big[1 - \beta \big(T-T_0\big)\Big] \]
		
		\paragraph{Compensazione tramite ponte di Wheatstone} In generale utilizzare l'auto-compensazione per ridurre l'effetto interferenze determina risultati accettabili solamente per intervalli di temperatura nell'intorno della condizione nominale dell'estensimetro. 
		
		Un modo per compensare gli effetti termici è dunque quello di effettuare la correzione direttamente sul ponte di Wheatstone. Infatti considerando il caso di un quarto di ponte (dove dunque le altre 3 resistenze sono costanti), la resistenza variabile associata all'estensimetro provocherà uno sbilanciamento del ponte dovuto sia all'effetto di deformazione del materiale, sia dovuto al segnale interferente dell'effetto termico:
		\[ \frac{\Delta e_0}{E} \approx \frac 1 4 K \big(\epsilon_1 + \epsilon_{1\Delta T}\big)  \]
		
		Questo sistema non porta alcuna compensazione, tuttavia se si considerasse un mezzo ponte di Wheatstone dove alla resistenza $R_1$ è associato un estensimetro collegato al materiale e a $R_2$ un estensimetro non collegato a nulla (seguendo lo schema del ponte a pag. \pageref{wheatstone}, figura \ref{wheatstone}), è possibile osservare come questo sistema compensi gli effetti termici:
		\[ \frac{\Delta e_0}E \approx \frac 1 4 K \big(\epsilon_1 - \epsilon_2\big) = \frac 1 4 K \big(\epsilon_{1a} + \epsilon_{1\Delta T} - \epsilon_{2\Delta T}\big) \simeq \frac 1 4 K \epsilon_{1a} \]
		
		Si osserva dunque che tramite questo sistema, considerando gli estensimetri come aventi le stesse proprietà, gli effetti interferenti termici vengono automaticamente compensati dal ponte di Wheatstone.
	
	\subsection{Isolamento elettrico}
		Di fondamentale importanza è che il supporto dell'estensimetro (e il collante sottostante) forniscano un'adeguato isolamento elettrico (ordine di migliaia di $M\Omega$) per evitare che circuiti esterni interferiscano con quello di lettura dell'estensimetro.
		
		Considerando infatti che l'estensimetro ha una resistenza $R$, mentre l'isolante ha una resistenza $R_i$, essendo esse in parallelo è possibile determinare la variazione di resistenza \textit{fittizia} $\Delta R$ utilizzando l'espressione del partitore di tensione:
		\[ \Delta R = \frac{R \,R_i}{R+R_i} - R = - \frac R {1 + R_i/R}  \] 
		
		Considerando un caso in cui la resistenza iniziale dell'estensimetro sia $R=120\Omega$ e quella dell'isolante sia $R_i = 1M\Omega$, effettuando il calcolo si ottiene che
		\[ \frac{\Delta R}{R} = - 1.19\cdot 10^{-4} \qquad \Rightarrow \quad \epsilon = \frac{\Delta R / R}{K} \simeq -60 \frac{\mu m}{m}  \]
		In questo caso si osserva dunque che l'isolamento elettrico introduce un disturbo di circa il $6\%$ su una misura di $1000 \mu m/m$.
		
		Considerando di aumentare la resistenza dell'isolante ad un valore $R_i = 1000 M\Omega$ allora si osserva che il disturbo dovuto alla variazione di resistenza è trascurabile in quanto di circa $0.006\%$ su una misura di $1000 \mu m / m$:
		\[ \frac{\Delta R}{R} = -1.19\cdot 10^{-7} \qquad \Rightarrow \quad \epsilon= \frac{\Delta R/R}{K} \simeq -0.06 \frac{\mu m}{m}  \]
		
	\subsection{Estensimetri a semiconduttore}
		Gli \de{estensimetri a semiconduttore} invece che essere composti da una grigli di materiale conduttore sono rappresentati da una lamina di semiconduttore con drogaggio $p$ o $n$.
		
		Questi estensimetri hanno un'elevata sensibilità piezoresistiva e quindi una sensibilidenominatoretà $K$ più elevata (rispetto agli estensimetri a conduttore a filo) a scapito di una maggiore sensibilità alla temperatura e ad un'elevata fragilità meccanica.
		
		In generale la variazione della resistenza $\Delta R/R$ in funzione dell'estensione $\epsilon$ è approssimata ad una funzione polinomiale di parametri $c_i$ dipendenti dalla temperatura del tipo:
		\[ \frac{\Delta R}{R} = c_1\epsilon + c_2 \epsilon^2 + c_3 \epsilon^3  \]
		
\section{Dinamometri}
		
		
		
		
		
		
		
		
		
		
		
		
		
		
		
		
		
		
		
		
		
		
		
		
		
		
		
		
		