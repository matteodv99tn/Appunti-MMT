\chapter{Termocoppie e termoresistenze}
	La misura di temperatura avviene misurando la differenza di forza elettromotrice (tensione) di una termocoppia
	
	\paragraph{Effetto Peltier} Quando una corrente inizia a scorrere nella termocoppia, sui giunti metallici si genera del calore (o esso viene dissipato) in modo da controbilanciare l'effetto Seebeck, anche se spesso questo effetto viene trascurato.
	
	\paragraph{Effetto Thomson} L'effetto Thomson rileva la dispersione di calore dovuta al filo che collega le due termocoppie.
	
	In generale
	
	\[ fem_{termocoppia} = \underbrace{c_1 \big(T_1-T_2\big)}_\textrm{Seebeck} +\underbrace{ c_2\big(T_1^2-T_2^2\big)}_\textrm{Thomson}\]
	dove $c_1,c_2$ sono due costanti generalmente fornite dal produttore ma che devono essere tarate per migliorare la precisione di misura. In generale i fili di connessione delle due giunzioni possono essere esposti a temperature ambientali variabili in quanto non influiscono nella misurazione della forza elettromotrice della termocoppia.
	
	Nella pratica le termocoppie sono caratterizzate dal non avere la necessità di avere un ambiente termostatato, ma si integra il controllore con un sensore di temperatura con il quale fare il confronto di temperatura; il sistema si comporta come un termometro dunque.
	
	
\section{Termoresistenze}