\chapter{Esercizi d'esame}

\section{Problemi meccanici alle impedenze generalizzate}
	
	\subsection*{28 luglio 2005}
		\paragraph{Testo} Dato lo schema di un accelerometro estensimetrico in \textbf{figura}; nel sistema l'ingresso è data dall'accelerazione $a$ che è il misurando, mentre l'uscita del sistema è lo sbilanciamento $\Delta V$ del ponte di Wheatstone. 
		
		\textbf{FIGURA}
	
		All'interno della scatola del sistema è presente una massa $m = 50g$ il cui moto è smorzato (con un valore di smorzamento $c_{eq} = 0.2 kg/s$) da un bagno d'olio che circonda il sistema. Il principio di misura si basa sulla misura di deformazione (con coefficiente elastico $k_{eq} = 10kN/m$) della mensola su cui si poggia la massa in funzione dell'accelerazione applicata.
		
		\paragraph{Risoluzione} Imponendo al sistema di misura un'accelerazione $a$ positiva, allora la massa per il terzo principio della dinamica spingerà \textit{verso il basso} con una forza $F = ma$: questo porta ad una deflessione \textit{a S} della mensola di appoggio.
		
		Supponendo di aver posto 2 estensimetri per ogni estremità della mensola dove si attacca con la base, allora l'estensimetro superiore tenderà ad allungarsi (aumento di resistenza), mentre quello inferiore tenderà a contrarsi (diminuzione di resistenza)
		
		
		
		
		